\section{Widman ``holes filling technique''}

A sharp version of the Caccioppoli-Leray inequality~\eqref{CLI} has been proven by \underline{Widman}.\\
We can illustrate that in the simple case of \(f=0, F=0\).\\
Observe that, with the notation of the~\eqref{CLI} proof, since \(\abs{\nabla u}\leq \frac{4}{R} \chi_{B_{R}\setminus B_{\sfrac{R}{2}}}\) one obtains
\begin{gather}
	\int\limits_{B_{\sfrac{R}{2}}}^{} \abs{\nabla u(x)}^{2} \dd{x} leq \frac{c}{R^{2}} \int\limits_{B_{R} \setminus B_{\sfrac{R}{2}}}^{} \abs{u(x)-k}^{2} \dd{x} \label{001}
\end{gather}
for some positive constant \(c\) independent of \(R\).\\
Now the idea is to choose \(\kappa := \fint_{B_{R}\setminus B_{\sfrac{R}{2}}} u(x) \dd{x}\) so that we can estimate the r.h.s of~\eqref{001} using the Poincarè inequality with explicit scaling, i.e.
\begin{gather}
	\int\limits_{B_{R}\setminus B_{\sfrac{R}{2}}}^{} \abs{u(x)-\fint\limits_{B_{R}\setminus B_{\sfrac{R}{2}}}^{} u \dd{x} }^{2} \dd{x} \leq cR^{2}\int\limits_{B_{R}\setminus B_{\sfrac{R}{2}}}^{} \abs{\nabla u(x)}^{2} \dd{x}
\end{gather}
to get
\begin{align}
	                       & \int\limits_{B_{\sfrac{R}{2}}}^{} \abs{\nabla u(x)}^{2} \dd{x} \leq c \int\limits_{B_{R}\setminus B_{\sfrac{R}{2}}}^{} \abs{\nabla u(x)}^{2} \dd{x}  \\
	\Leftrightarrow  (c+1) & \int\limits_{B_{\sfrac{R}{2}}}^{} \abs{\nabla u(x)}^{2} \dd{x} \leq c \int\limits_{B_{R}}^{} \abs{\nabla u(x)}^{2} \dd{x}
\end{align}
Setting \( \vartheta := \frac{c}{c+1}<1\) we get
\begin{gather}
	\int\limits_{B_{\sfrac{R}{2}}}^{} \abs{\nabla u(x)}^{2} \dd{x} \leq \vartheta \int\limits_{B_{R}}^{} \abs{\nabla u(x)}^{2} \dd{x}
\end{gather}
\underline{Iterating} the previous estimates \(d\) times for radii
\begin{gather}
	2^{1}r \to 2^{2}r \to 2^{3}r \to \dots \to 2^{d}r
\end{gather}
and choosing \(r\) such that
\begin{gather}
	2^{d}r < R< 2^{d+1}r \label{002}
\end{gather}
we get
\begin{gather}
	\int\limits_{B_{R}}^{} \abs{\nabla u}^{2} \dd{x} \leq \vartheta^{d}\int\limits_{B_{R}}^{} \abs{\nabla u}^{2} \dd{x}
\end{gather}
Setting \(\alpha \log_{2}(\sfrac{1}{\vartheta})\), i.e. \(\vartheta = \sfrac{1}{2^{\alpha}}\) we have
\begin{gather}
	\vartheta^{d}= \frac{1}{2^{\alpha d}} = {\Big( \frac{1}{2^{d}} \Big)}^{\alpha} \overset{\eqref{002}}{\leq} 2^{\alpha}{\Big( \frac{r}{R} \Big)}^{\alpha}.
\end{gather}
Hence,
\begin{gather}
	\int\limits_{B_{r}}^{} \abs{\nabla u}^{2} \dd{x} \leq 2^{\alpha} {\Big( \frac{r}{R} \Big)}^{\alpha} \int\limits_{B_{R}}^{} \abs{\nabla u}^{2} \dd{x}.
\end{gather}
For \(n=2\) the estimate above implies \(u \in C^{0, \sfrac{\alpha}{2}}(\Omega; \mathbb{R}^{m})\).\\
In fact the idea that the decay of the \(L^{p}\)-norm of the gradient is related to its Hölder continuity will play a crucial role in the rest of the course, and we will discuss in detail in the next lectures.

\section{Continuity via embedding}

The Sobolev embedding theorem for \(W^{1,p}(\Omega; \mathbb{R})\) says that
\begin{align}
	\begin{cases}
		p<n \qquad   & W^{1,p}(\Omega; \mathbb{R}) \hookrightarrow L^{p^{\star}}(\Omega;\mathbb{R}) \quad \text{continuously} \quad p^{\star}=\frac{np}{n-p} \\
		p=n \qquad   & W^{1,n}(\Omega; \mathbb{R}) \hookrightarrow L^{q^{\star}}(\Omega;\mathbb{R}) \quad \text{compactly} \quad \forall 1 \leq q < \infty   \\
		p > n \qquad & W^{1,p}(\Omega; \mathbb{R}) \hookrightarrow C^{0, 1-\sfrac{n}{p}}(\Omega;\mathbb{R}) \quad \text{continuously}
	\end{cases}
\end{align}
Hence a way to prove continuity of a Sobolev function is to prove that it belongs to \(W^{1,p}\) for \(p>n\).

\section{Embedding for higher order Sobolev spaces}

We recall that higher order Sobolev spaces \(W^{k,p}(\Omega; \mathbb{R})\) with \(k \geq 1\) integer and \(1 \leq p \leq \infty \) are recursively defined as
\begin{gather}
	W^{k,p}(\Omega, \mathbb{R}):= \{ u \in W^{1,p}(\Omega; \mathbb{R}): \nabla u \in W^{k-1,p}(\Omega; \mathbb{R}^{n})\}.
\end{gather}
Another way to prove continuity, applicable if \(p < n\), is to use \(W^{k,p}\) for \(k\) large enough. In fact, it holds true that:
\begin{enumerate}[label= (\arabic*)]
	\item if \(kp<n\) then \(W^{k,p}(\Omega; \mathbb{R})\hookrightarrow L^{p}(\Omega;\mathbb{R})\) for all \(1 \leq q \leq p_{k}^{\star}\), when \(p_{k}^{\star}=\frac{np}{n-kp}\)
	\item if \(kp=n\) then \(W^{k,p}(\Omega; \mathbb{R}) \hookrightarrow L^{q}(\Omega; \mathbb{R}) \) for all \( 1 \leq q\leq \infty \)
	\item if \( kp > n \) and \( k-\frac{n}{p}\notin \mathbb{N}, W^{k,p}(\Omega;\mathbb{R}) \hookrightarrow C^{l,\alpha}(\overline{\Omega; \mathbb{R}})\) for \( l= \left\lfloor k-\frac{n}{p} \right\rfloor  \) and \( 0 \leq \alpha\leq k-\frac{n}{p}-l \)
	\item if \( kp> n \) and \( k-\frac{n}{p}=l+1 \in \mathbb{N}, W^{k,p}(\Omega; \mathbb{R}) \hookrightarrow C^{l,\alpha} (\overline{\Omega}; \mathbb{R})\) for all \( 0 \leq \alpha < 1 \)
\end{enumerate}

\section{A priori estimates and the Nirenberg method}

If \( u \in  H_{loc}^{1} (\Omega; \mathbb{R})  \) (for the moment we are not interested at the behavior of \( u \) at \( \partial \Omega \)) is a weak solution of a system of elliptic PDEs we cannot apply previous remark to prove classical regularity, i.e.\ differentiability of \( u \) without assuming existence and some integrability of higher order weak derivatives of \( u \). In fact the previous remark is not really exploiting the equation.\\
\par
\underline{How to gain better integrability?}\\
What   follows goes under the name of \underline{Nirenberg's method}.\\
Let us consider the simplest setting and consider a solution \( u \in H_{loc}^{1} (\Omega)  \) of the Poisson equation
\begin{gather}
	- \Delta u = f, \qquad f \in L_{loc}^{2} (\Omega ;\mathbb{R})
\end{gather}
We want to prove that \( u \in H_{loc}^{2} (\Omega ; \mathbb{R}) =W_{loc}^{2,2} (\Omega ;\mathbb{R})  \), as this will be the first step to transfer regularity information from the data \( f \) to the solution \( u \).\\
\\
Let us start with supposing that we already knew that \( \partial_{x_{\alpha }} u \in H_{loc}^{1} (\Omega ;\mathbb{R})  \), then we know that
\begin{gather}
	-\Delta (\partial_{x_{\alpha }} u) = \partial_{x_{\alpha }} f \qquad \text{in a weak sense}
\end{gather}
To check it, test \( -\Delta u=f \) with \( \partial_{x_{\alpha }} \varphi  \) and integrate by parts to get
\begin{align}
	\int \nabla u \nabla (\partial_{x_{\alpha }}\varphi )  \dd{x}           & = \int f \partial_{x_{\alpha }}\varphi  \dd{x}  \\
	\overset{=}{\int \nabla \partial_{x_{\alpha }}(\nabla \varphi)  \dd{x}} & \overset{I.P.}{=} - \int \underbrace{\partial_{x_{\alpha }} \Delta u}_{\Delta (\partial_{x_{\alpha }} u) } \cdot \Delta \varphi \dd{x} = - \int \Delta (\partial_{x_{\alpha }}u) \Delta \varphi  \dd{x}
\end{align}
Weak derivatives commute:
\begin{gather}
	\int \partial_{1}\partial_{2} u \varphi  \dd{x} \overset{I.P.}{=} \int u \partial_{1} \partial_{2} \varphi  \dd{x} \overset{\varphi \text{ is regular}}{=} \int u \partial_{2}\partial_{1} \varphi \dd{x} \overset{I.P.}{=} \int \partial_{2}\partial_{1}u \varphi \dd{x}
\end{gather}
Hence \( \int \nabla (\partial_{x_{\alpha }}u) \nabla \varphi  \dd{x} = -\int f \partial_{x_{\alpha }}\varphi  \dd{x} \) or \( -\nabla (\partial_{x_{\alpha }}u) = \partial_{x_{\alpha}}f \).\\
Hence, for every ball \( B_{R} (x_{0}) \ssubset \Omega  \) (\( \overline{B_{R} (x_{0})} \subset \Omega  \)) we use the Caccioppoli-Leray inequality~\eqref{CLI}  to get:
\begin{gather}
	C_{CL} \int\limits_{B_{\sfrac{R}{2}}(x_{0})}^{} \abs{\nabla (\partial_{x_{\alpha }}u) }^{2} \dd{x} \leq \frac{1}{R^{2}} \int\limits_{B_{R}(x_{0})}^{} \abs{\partial_{x_{\alpha }}U(x) }^{2} \dd{x} + R^{2} \int\limits_{B_{R}(x_{0})}^{} \abs{f(x) }^{2} \dd{x} \qquad \text{\eqref{CLI}}
\end{gather}
that provides an explicit bound on the \( H_{loc}^{2} \) norm of \( u \) in terms of its \( H^{1} \) norm.\\
\par
In the previous discussion we have considered \( u \in H_{loc^{1}}(\Omega ; \mathbb{R})  \) to be a solution of the Poisson equation \( -\nabla u = f \) in \( \Omega  \) with \( f \in L^{2}(\Omega ) \). Assuming \( u\in H_{loc}^{2} \) we have obtained the following Caccioppoli-Leray estimate:
\begin{gather}
	C_{CL} \int\limits_{B_{\sfrac{R}{2}}(x_{0})}^{} \abs{\nabla (\partial_{x_{\alpha }}u) }^{2} \dd{x} \leq \frac{1}{R^{2}} \int\limits_{B_{R}(x_{0})}^{} \abs{\partial_{x_{\alpha }}U(x) }^{2} \dd{x} + R^{2} \int\limits_{B_{R}(x_{0})}^{} \abs{f(x) }^{2} \dd{x} \qquad \text{\eqref{CLI}}
\end{gather}
Can we remove the \enquote{a priori} regularity assumption?\\
Here, for the Poisson equation, it is simple.\\
Consider the convolution \( u \ast \rho_{\varepsilon } \). Since \( -\Delta  u = f \) we have \( -\Delta  (u \ast \rho_{\varepsilon }) = f\ast \rho _{\varepsilon }\).\\
Now observe that
\begin{align}
	C_{CL} \int\limits_{B_{\sfrac{R}{2}}}^{} \abs{\nabla (\partial_{x_{\alpha }}u\ast \rho _{\varepsilon })}*2 \dd{x} & \leq \frac{1}{R^{2}}\int\limits_{B_{R}}^{} \abs{\partial_{x_{\alpha }}u \ast \rho _{\varepsilon }}^{2} \dd{x} + R^{2}\int\limits_{B_{R}}^{} \abs{f\ast \rho _{\varepsilon }}^{2} \dd{x}  \\
	                                                                                                                  & \leq  \frac{1}{R^{2}} \int\limits_{B_{R}}^{} \abs{\partial_{x_{\alpha }}u}^{2} \dd{x} + R^{2} \int\limits_{B_{R}}^{} \abs{f}^{2} \dd{x}
\end{align}
As a result \( \forall \alpha ,\beta \, \norm{\partial_{x_{\alpha }}\partial_{x_{\beta}}(u\ast \rho _{\varepsilon })}_{L_{loc}^{2}} \leq  C \). This means that, up to subsequences \( \partial_{x_{\alpha }}\partial_{x_{\beta}}(u\ast \rho _{\varepsilon }) \myxrightharpoonup[\varepsilon  \to 0]{L^{2}} g\). Since \( \partial_{x_{\alpha }}(u\ast \rho _{\varepsilon }) \myxrightarrow[]{L^{2}} \partial_{x_{\alpha }} u \), we have that \( g = \partial_{x_{\alpha }}\partial_{x_{\beta}}u\) and that the whole sequence \( \partial_{x_{\alpha }}\partial_{x_{\beta}}(u\ast \rho _{\varepsilon }) \) converges to \( \partial_{x_{\alpha }}\partial_{x_{\beta}}u \). \\
As a result \( \norm{\partial_{x_{\alpha }}\partial_{x_{\beta}}u}_{L_{loc}^{2}} \leq  C \) and \( u \in H_{loc}^{2} (\Omega )  \).\\
\par

We have used the following result:
\begin{lem}[Stability of weak derivatives]
	\( u_{k}\in W^{1,p}(\Omega )\) for some \( 1 < p < \infty, \forall k \in  \mathbb{N} \). Assume \( u_{k} \myxrightarrow[]{} u  \) in \( L^{p} \) and \( \sup_{k}\norm{\nabla u_{k}}_{p}\leq C \), then \( u\in W^{1,p}(\Omega ) \) and \( \nabla u_{k} \myxrightharpoonup[]{} \nabla u \) in \( L^{p} \).
\end{lem}
The same idea does not work so easily when the coefficients \( A_{ij}^{\alpha \beta } \) are not constant. In fact in this case differentiating the equation produces \enquote{extra terms}.\\
Nirenberg's idea is to use difference quotients instead of derivatives. We introduce the notation
\begin{gather}
	\Delta _{h, \alpha }u(x) = \frac{u(x+he_{\alpha})-u(x)}{h} =: \frac{\tau _{h, \alpha }U(x)-u(x) }{h }
\end{gather}
The following properties can be checked to hold true:
\begin{enumerate}[label= \( \bullet \)  ]
	\item Discrete Leibniz rule
	      \begin{align}
		      \Delta _{h, \alpha }(uv)
		       & = (\tau _{h, \alpha } u) \Delta _{h,\alpha }v + (\Delta _{h, \alpha} u) v  \\
		       & =    (\tau _{h, \alpha } v) \Delta _{h,\alpha }u + (\Delta _{h, \alpha} v) u
	      \end{align}

	\item Integration by parts rule
	      \begin{gather}
		      \int\limits_{\Omega}^{} \varphi (x) \Delta _{h, \alpha } u(x)   \dd{x} = -\int\limits_{\Omega}^{} u(x) \Delta _{-h, \alpha }\varphi (x)    \dd{x}
	      \end{gather}
	      for all \( \varphi \in  C_{c}^{1} (\Omega ; \mathbb{R}), \abs{h}< \operatorname{dist}(\operatorname{spt} \varphi, \partial \Omega )  \)
\end{enumerate}
The following lemma provides a characterization of \( W^{1,p} \) functions with \( p>1 \), in terms of uniform \( L^{p} \) bounds of the corresponding discrete partial derivatives.

\begin{lem}[]\label{lemma01}
	Consider \( u\in L_{loc}^{p}(\Omega ;\mathbb{R}) \), with \( 1< p \leq  \infty  \) and fix \( \alpha \in \{ 1,2,\ldots,n \} \). The partial derivative \( \partial_{x_{\alpha }}u \) belongs to \( L_{loc}^{p}(\Omega ;\mathbb{R}) \) if and only if the family \( \Delta _{h,\alpha }u \) is uniformly bounded in \( L_{loc}^{p} \) as \( h \to 0 \).\\
	More precisely, if \( \forall \Omega' \ssubset \Omega \, \exists C=C(\Omega') \) such that
	\begin{gather}
		\abs{\int\limits_{\Omega'}^{} (\Delta _{h,\alpha }u) \varphi  \dd{x}}\leq c\norm{\varphi }_{L^{p'}(\Omega';\mathbb{R})} \qquad \varphi \in C_{c}^{1}(\Omega';\mathbb{R})
	\end{gather}
	with \( \frac{1}{p}+\frac{1}{p'}=1 \) and \( \abs{h}< \frac{1}{2} \operatorname{dist}(\Omega';\partial \Omega )  \).
\end{lem}
We now see how the previous lemma allows us to obtain regularity. We stick to the Poisson equation for the moment. \par
Suppose \( f \in H_{loc}^{1}(\Omega ;\mathbb{R})  \) and \( -\Delta u=f \) for some \( u \in H_{loc}^{1}(\Omega ; \mathbb{R})  \).\\
Being the equation translation invariant, we can write \( -\Delta (\tau _{h,\alpha }u) = \tau _{h, \alpha }f \), hence \( -\Delta (\Delta _{h, \alpha }u) = \Delta _{h, \alpha }f \) for any \( \Omega' \ssubset \Omega  \) and \( \abs{h}< \operatorname{dist}(\Omega', \partial \Omega )  \). \\
By Lemma~\ref{lemma01} it holds \( \Delta _{h, \alpha }f \) is bounded in \( L_{loc}^{2} \) uniformly in \( h \). By~\eqref{CLI} \( \abs{\nabla \Delta _{h, \alpha }U} \) is bounded in \( L_{loc}^{2}(\Omega ;\mathbb{R})  \), thanks to the Lemma~\ref{lemma01} (applied componentwise) we have that
\begin{gather}
	\partial_{x_{\alpha }}(\nabla u) \in L_{loc}^{2}(\Omega ;\mathbb{R}^{n})
\end{gather}

That is, by the arbitrariness of \( \alpha \in \{ 1,2,\ldots,n \} \), \( u \in H_{loc}^{2}(\Omega ;\mathbb{R})  \). We are left to prove Lemma~\ref{lemma01}.\\
\par
We now state and prove the first interior regularity theorem.

\begin{thm}[\(H^{2}\)-regularity]
	Let \( \Omega  \) be an open domain in \( \mathbb{R} \). Consider a map \( A\in C_{loc}^{0,1}(\Omega ; \mathbb{R}^{m^{2}\times n^{2}})  \)
	such that \( A(x) := A_{ij}^{\alpha \beta }(x)\) satisfies the Legendre-Hademard condition~\eqref{LH} for some continuous and positive ellipticity function \( \lambda : \Omega \to \mathbb{R}\),
	as well as the uniform bound
	\[ \sup_{x \in \Omega } \abs{A_{ij}^{\alpha \beta }(x)}\leq \Lambda < \infty. \] Then,
	for every \( u \in H_{loc}^{1}(\Omega ; \mathbb{R}^{m})  \) weak solution of the equation \[ - \sum\limits_{\alpha ,\beta ,j}^{}\partial_{x_{\alpha }}(A_{ij}^{\alpha \beta }\partial_{x_{\beta}}u^{j}) = f_{i}-\sum\limits_{\alpha}^{} \partial_{x_{\alpha }}F_{i}^{\alpha } \qquad i=1,2,\ldots,m  \]
	with data \( f \in L_{loc}^{2}(\Omega ;\mathbb{R}^{m}) \) and \( F \in H_{loc}^{1}(\Omega ;\mathbb{R}^{m \times n})  \), one  has that \( u \in H_{loc}^{2}(\Omega ; \mathbb{R}^{m})  \) and for every \( \Omega' \ssubset \Omega'' \ssubset \Omega \) there exists \( c:=c(\Omega', \Omega'', A)  \) such that
	\[ \int\limits_{\Omega'}^{} \abs{\nabla ^{2}u}^{2} \dd{x} \leq c \left( \int\limits_{\Omega''}^{} \abs{u}^{2} \dd{x} + \int\limits_{\Omega''}^{} \abs{f}^{2} \dd{x} + \int\limits_{\Omega''}^{} \abs{\nabla F}^{2} \dd{x} \right) \]
\end{thm}

\begin{remark}[]
	Even if we have stated the theorem for a generic \( \Omega' \ssubset \Omega  \), it is enough to prove it for balls inside \( \Omega  \).
	More precisely. It is enough to prove it for balls \( B_{R}(x_{0}) \) where \( x_{0}\in \Omega' \) and \( R < \frac{1}{2}\overset{dist}(\Omega', \partial \Omega )  \).
	\begin{center}
		\includegraphics[scale=0.45]{pictures/picture02.png}
	\end{center}
	The general result can then be obtained by a compactness and covering argument (Exercise).\\
	For the case of a ball we need to prove that
	\[ \int\limits_{B_{\sfrac{R}{2}(x_{0}) }}^{} \abs{\nabla ^{2}u}^{2} \dd{x} \leq c \left( \int\limits_{B_{2R}(x_{0})}^{} \abs{u}^{2} \dd{x} + \int\limits_{B_{2R}(x_{0})}^{} \abs{f}^{2} \dd{x} + \int\limits_{B_{2R}(x_{0})}^{} \abs{\nabla F}^{2} \dd{x} \right) \] for every \( x_{0} \in \Omega' \).
\end{remark}

\begin{proof}
	\( \bullet \)  Assume w.l.o.g that \( x_{0}=0 \) and \( F=0 \). (note that the term \( \sum_{\alpha }^{}\partial_{x_{\alpha }}F_{i}^{\alpha } \) can always be absorbed into \( f \). In fact \( \norm{f+ \operatorname{div}F^{i}}_{2}\leq \norm{f}_{2}+\norm{\nabla F}_{2} \))\\
	\( \bullet \)  Moreover we assume that \( \lambda \) is constant (it is possible to reduce to the general case, see next lectures) \\
	We start observing that the equation in its weak formulation reads as
	\[ \int\limits_{\Omega}^{} \left\langle A \nabla u, \nabla \varphi  \right\rangle  \dd{x} = \int\limits_{\Omega}^{} \left\langle f,\varphi  \right\rangle  \dd{x}, \qquad \forall  \varphi \in  C_{c}^{\infty }(\Omega ; \mathbb{R}) \]
	In order to simplify the notation in the proof we let \( e_{\gamma } \) be a fixed vector and set \( \tau _{h}:= \tau _{h, \gamma  } \) and \( \Delta _{h := \Delta _{h, \gamma }} \).\\
	We take as test function \( \tau _{-h}\varphi  \), for \( h \) small enough and change variables to get
	\[ \int\limits_{\Omega}^{} \left\langle \tau _{h}(A \nabla u), \nabla \varphi  \right\rangle  \dd{x} = \int\limits_{\Omega}^{} \left\langle \tau _{h}f, \varphi  \right\rangle  \dd{x} \]
	subtracting the two previous equations anf dividing by \( h \) we have that (using Leibniz)
	\begin{align}
		  & \int\limits_{\Omega}^{} \frac{1}{h} \left[ \left\langle A \nabla u, \nabla \varphi  \right\rangle  \right]- \left\langle \tau _{h}(A \nabla u), \nabla \varphi  \right\rangle \dd{x}  \\
		= & \int\limits_{\Omega}^{} \left\langle \underbrace{A \nabla u - \tau _{h}(A \nabla u)}_{h}, \nabla \varphi  \right\rangle  \dd{x}  \\
		= & \int\limits_{\Omega}^{} \left\langle \Delta _{h}(A \nabla u), \nabla \varphi  \right\rangle  \dd{x}  \\
		= & \int\limits_{\Omega}^{} \left\langle \tau _{h} A \nabla (\Delta _{h} u), \nabla \varphi \right\rangle  + \left\langle (\Delta _{h}A) \nabla u, \nabla \varphi  \right\rangle \dd{x}  \\
		= & \int\limits_{\Omega}^{} \left\langle \Delta _{h}f, \varphi  \right\rangle  \dd{x},
	\end{align}
	i.e,
	\[ \int\limits_{\Omega}^{} \left\langle (\tau _{h}A) \nabla (\Delta _{h}u)  \right\rangle  \dd{x} = \int\limits_{\Omega}^{} \left\langle \Delta _{h} f, \varphi  \right\rangle  \dd{x} - \int\limits_{\Omega}^{} \left\langle (\Delta _{h}A) \nabla u, \nabla \varphi \right\rangle  \dd{x} \]
	This is the weak formulation of the equation
	\[ - \sum\limits_{\alpha ,\beta ,j}^{} \partial_{x_{\alpha}} \left( {(\tau _{h}A)}_{ij}^{\alpha \beta } \partial_{x_{\beta }} v^{j} \right)= f_{i}' - \sum\limits_{\alpha }^{} \partial_{x_{\alpha }} G_{i}^{\alpha }, \qquad i=1,2,\ldots,m \tag{EQ} \label{EQ} \]
	where \( v= \Delta _{h}u \). \( f'=\Delta _{h}f \) and \( G=-(\Delta _{h}A) \nabla u \). The basic idea now is to use~\eqref{CLI}. A direct application of it would lead to an estimate in terms of the \( L^{2} \) norm of \( f'=\Delta _{h}f \) which we know can be uniformly bounded in \( h \) only if \( f \in H_{loc}^{1}(\Omega )  \) (by the characterization of Sobolev spaces in terms of difference quotients). Since we have only assumed \( f \in L_{loc}^{2}(\Omega) \), we need to proceed carefully and \enquote{adapt} the proof of~\eqref{CLI}. We consider the cut-off function \( \eta \) compactly supported in \( B_{R} \), \( \eta \in [0,1]  \), \( \eta \equiv 1 \) on \( B_{\sfrac{R}{2}} \) and \( \abs{\nabla \eta }\leq \sfrac{4}{R} \). We need test~\eqref{EQ} with \( \varphi := \eta ^{2}\delta _{h}u= \eta ^{2}v \), where \( \abs{h}<\sfrac{R}{2} \). \\
	As in the proof of~\eqref{CLI} we get
	\[ \frac{3 \lambda }{4}\int\limits_{B_{R}}^{} \eta ^{2}\abs{\nabla v}^{2} \dd{x} \leq  \frac{4 \Lambda \varepsilon }{R}\int\limits_{B_{R}}^{} \eta ^{2}\abs{\nabla v}^{2} \dd{x}+ \left( \frac{4 \Lambda }{R \varepsilon }+ \frac{4}{R^{2}}\right) \int\limits_{B_{R}}^{} \abs{v}^{2} \dd{x}	+ \int\limits_{B_{R}}^{} \eta ^{2}v \Delta _{h}f \dd{x} + \left( \frac{1}{\lambda } + 4	 \right) \int\limits_{B_{R}}^{} \abs{G}^{2} \dd{x}. \]
	Choosing \( \varepsilon > 0 \) we absorb \( \frac{4 \Lambda \varepsilon }{R} \int_{B_{R}}^{} \eta ^{2} \abs{\nabla v}^{2} \dd{x} \) in the L.H.S\ and we get that for some constant \( c=c(\lambda , \Lambda , R) > 0 \)
	\begin{gather}
		c \int\limits_{B_{R}}^{} \eta ^{2}\abs{\nabla v}^{2} \dd{x} \leq \underbrace{\int\limits_{B_{R}}^{} \abs{v}^{2} \dd{x}}_{\eqref{004}} + \underbrace{\int\limits_{B_{R}}^{} \eta ^{2} v \Delta _{h}f \dd{x}}_{\eqref{005} } + \underbrace{\int\limits_{B_{R}}^{} \abs{G}^{2} \dd{x}}_{\eqref{006}} \label{003}
	\end{gather}
	We consider the different terms separately. We notice that (see~\eqref{001} in the proof of Lemma~\ref{lemma01})
	\begin{gather}
		\int\limits_{B_{R}}^{} \abs{v}^{2} \dd{x} = \int\limits_{B_{R}}^{} \abs{\Delta _{h} u}^{2} \dd{x} \leq \int\limits_{B_{R+h}}^{} \abs{\nabla u}^{2} \dd{x} \label{004}
	\end{gather}
	The R.H.S\ of the inequality above can be estimated by the~\eqref{CLI}. In fact \( \int_{B_{R+h}}^{} \abs{\nabla u}^{2} \dd{x} \leq \int_{B_{\sfrac{3}{2}R}}^{} \abs{\nabla u}^{2} \dd{x} \) which can be in turn be estimated by~\eqref{CLI} for \( u \) between the balls \( B_{\sfrac{3R}{2}} \) and \( B_{2R} \), with an upper bound of the type we are looking for. Concerning the term~\eqref{2} we have
	\begin{align}
		\abs{\int\limits_{B_{R}}^{} \eta ^{2} v \Delta _{h} f \dd{x}} \overset{\text{discrete by I.P.}}{=} & \abs{\int\limits_{B_{R}}^{} - \Delta _{-h} (\eta ^{2} v) f  \dd{x}}  \notag  \\
		\overset{\text{Young }p=q=2}{\leq}                                                                 & \widetilde{\varepsilon} \int\limits_{B_{R}}^{} \abs{\Delta _{-h}(\eta ^{2}v)}^{2} \dd{x} + \frac{1}{ \widetilde{\varepsilon}} \int\limits_{B_{R}}^{} \abs{f}^{2} \dd{x} \label{004}
	\end{align}
	The term \( \int_{B_{R}}^{} \abs{f}^{2} \dd{x} \leq \int_{B_{2R}}^{} \abs{f}^{2} \dd{x} \) is already fine for the estimate we want. For the other term we have

\end{proof}
