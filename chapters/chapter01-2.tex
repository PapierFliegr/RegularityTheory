\section{Widman ``holes filling technique''} 

A sharp version of the Caccioppoli-Leray inequality~\eqref{CLI} has been proven by \underline{Widman}.\\
We can illustrate that in the simple case of \(f=0, F=0\).\\
Observe that, with the notation of the~\eqref{CLI} proof, since \(\abs{\nabla u}\leq \frac{4}{R} \chi_{B_{R}\setminus B_{\sfrac{R}{2}}}\) one obtains
\begin{gather}
    \int\limits_{B_{\sfrac{R}{2}}}^{} \abs{\nabla u(x)}^{2} \dd{x} leq \frac{c}{R^{2}} \int\limits_{B_{R} \setminus B_{\sfrac{R}{2}}}^{} \abs{u(x)-k}^{2} \dd{x} \label{001}
\end{gather}
for some positive constant \(c\) independent of \(R\).\\
Now the idea is to choose \(\kappa := \fint_{B_{R}\setminus B_{\sfrac{R}{2}}} u(x) \dd{x}\) so that we can estimate the r.h.s of~\eqref{001} using the Poincarè inequality with explicit scaling, i.e. 
\begin{gather}
    \int\limits_{B_{R}\setminus B_{\sfrac{R}{2}}}^{} \abs{u(x)-\fint\limits_{B_{R}\setminus B_{\sfrac{R}{2}}}^{} u \dd{x} }^{2} \dd{x} \leq cR^{2}\int\limits_{B_{R}\setminus B_{\sfrac{R}{2}}}^{} \abs{\nabla u(x)}^{2} \dd{x} 
\end{gather}
to get
\begin{align}
    & \int\limits_{B_{\sfrac{R}{2}}}^{} \abs{\nabla u(x)}^{2} \dd{x} \leq c \int\limits_{B_{R}\setminus B_{\sfrac{R}{2}}}^{} \abs{\nabla u(x)}^{2} \dd{x} \\
    \Leftrightarrow  (c+1) & \int\limits_{B_{\sfrac{R}{2}}}^{} \abs{\nabla u(x)}^{2} \dd{x} \leq c \int\limits_{B_{R}}^{} \abs{\nabla u(x)}^{2} \dd{x}
\end{align}
Setting \( \vartheta := \frac{c}{c+1}<1\) we get 
\begin{gather}
    \int\limits_{B_{\sfrac{R}{2}}}^{} \abs{\nabla u(x)}^{2} \dd{x} \leq \vartheta \int\limits_{B_{R}}^{} \abs{\nabla u(x)}^{2} \dd{x}
\end{gather}
\underline{Iterating} the previous estimates \(d\) times for radii
\begin{gather}
    2^{1}r \to 2^{2}r \to 2^{3}r \to \dots \to 2^{d}r
\end{gather}
and choosing \(r\) such that 
\begin{gather}
    2^{d}r < R< 2^{d+1}r \label{002}
\end{gather}
we get
\begin{gather}
    \int\limits_{B_{R}}^{} \abs{\nabla u}^{2} \dd{x} \leq \vartheta^{d}\int\limits_{B_{R}}^{} \abs{\nabla u}^{2} \dd{x}
\end{gather}
Setting \(\alpha \log_{2}(\sfrac{1}{\vartheta})\), i.e. \(\vartheta = \sfrac{1}{2^{\alpha}}\) we have
\begin{gather}
    \vartheta^{d}= \frac{1}{2^{\alpha d}} = {\Big( \frac{1}{2^{d}} \Big)}^{\alpha} \overset{\eqref{002}}{\leq} 2^{\alpha}{\Big( \frac{r}{R} \Big)}^{\alpha}.
\end{gather}
Hence, 
\begin{gather}
    \int\limits_{B_{r}}^{} \abs{\nabla u}^{2} \dd{x} \leq 2^{\alpha} {\Big( \frac{r}{R} \Big)}^{\alpha} \int\limits_{B_{R}}^{} \abs{\nabla u}^{2} \dd{x}.
\end{gather}
For \(n=2\) the estimate above implies \(u \in C^{0, \sfrac{\alpha}{2}}(\Omega; \mathbb{R}^{m})\).\\
In fact the idea that the decay of the \(L^{p}\)-norm of the gradient is related to its Hölder continuity will play a crucial role in the rest of the course, and we will discuss in detail in the next lectures.

\section{Continuity via embedding}

The Sobolev embedding theorem for \(W^{1,p}(\Omega; \mathbb{R})\) says that
\begin{align}
    \begin{cases}
         p<n \qquad &W^{1,p}(\Omega; \mathbb{R}) \hookrightarrow L^{p^{\star}}(\Omega;\mathbb{R}) \quad \text{continuously} \quad p^{\star}=\frac{np}{n-p}\\
        p=n \qquad  &W^{1,n}(\Omega; \mathbb{R}) \hookrightarrow L^{q^{\star}}(\Omega;\mathbb{R}) \quad \text{compactly} \quad \forall 1 \leq q < \infty \\
        p > n \qquad &W^{1,p}(\Omega; \mathbb{R}) \hookrightarrow C^{0, 1-\sfrac{n}{p}}(\Omega;\mathbb{R}) \quad \text{continuously}
    \end{cases}
\end{align}
Hence a way to prove continuity of a Sobolev function is to prove that it belongs to \(W^{1,p}\) for \(p>n\).

\section{Embedding for higher order Sobolev spaces}

We recall that higher order Sobolev spaces \(W^{k,p}(\Omega; \mathbb{R})\) with \(k \geq 1\) integer and \(1 \leq p \leq \infty \) are recursively defined as 
\begin{gather}
    W^{k,p}(\Omega, \mathbb{R}):= \{ u \in W^{1,p}(\Omega; \mathbb{R}): \nabla u \in W^{k-1,p}(\Omega; \mathbb{R}^{n})\}.
\end{gather}
Another way to prove continuity, applicable if \(p < n\), is to use \(W^{k,p}\) for \(k\) large enough. In fact, it holds true that:
\begin{enumerate}[label= (\arabic*)]
    \item if \(kp<n\) 
\end{enumerate}
