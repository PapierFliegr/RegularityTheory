\section{Decay estimates for systems with constant coefficients}

\begin{lem}[]
	Let \( A=A_{ij}^{\alpha \beta } \) be a constant matrix satisfying the Legendre-Hadamard condition~\eqref{LH} for some \( \lambda > 0 \), let \( \Lambda = \abs{A} \) and let \( u \in  H_{loc}^{1}(\Omega ; \mathbb{R}^{m})  \) satisfying the system
	\begin{gather}
		- \sum\limits_{\alpha, \beta, j}^{} \partial_{x_{\alpha }} \left( A_{ij}^{\alpha \beta } \partial_{x_{\beta}}  u^{j} \right) = 0, \qquad \forall  i \in \{ 1,2,\ldots , m \}
	\end{gather}
	Then for \( B_{r}(x_{0}) \subset  B_{R}(x_{0}) \ssubset \Omega  \) it holds
	\begin{align}
		\bullet & \int\limits_{B_{r}(x_{0})}^{} \abs{u}^{2} \dd{x} \leq c_{D} {\left( \frac{r}{R} \right)}^{n} \int\limits_{B_{R}(x_{0})}^{} \abs{u}^{2} \dd{x}  \\
		\bullet & \int\limits_{B_{r}(x_{0})}^{} \abs{u-u_{x_{0},r}}^{2} \dd{x} \leq c_{E} {\left( \frac{r}{R} \right)}^{n+2} \int\limits_{B_{R}(x_{0})}^{} \abs{u-u_{x_{0},R}}^{2} \dd{x}
	\end{align}
	with \( c_{D}=c_{D}(n,\lambda, \Lambda  ) \) and \( c_{E}=c_{E}(n, \lambda , \Lambda ) \), having used the notation
	\begin{gather}
		u_{x_{0},s}:= \frac{1}{\abs{B_{s}(x_{0})}} \int\limits_{B_{s}(x_{0})}^{} u(x)  \dd{x}
	\end{gather}
\end{lem}

\section{Regularity up to the boundary}

Let \( u \in H_{0}^{1}(\Omega ; \mathbb{R}^{m})  \) be a weak solution of
\begin{gather}
	- \sum\limits_{\alpha, \beta, j}^{} \partial_{x_{\alpha }} \left( A_{ij}^{\alpha \beta } \partial_{x_{\beta}}  u^{j} \right) = f_{i} - \sum\limits_{\alpha }^{} \partial_{x_{\alpha }} F_{i}^{\alpha }, \qquad i \in \{ 1,2,\ldots,m \}
\end{gather}
We make the following hypothesis: \\
\( f \in L^{2}(\Omega ; \mathbb{R}^{m}) \), \( F \in H^{1}(\Omega ; \mathbb{R}^{m \times n})  \), \( A \in C^{0,1}(\Omega ; \mathbb{R}^{m^{2}\times n^{2}})  \), \\
\( A(X)  \) satisfies the Legendre-Hadamard condition~\eqref{LH} uniformly with respect to \( x \in \Omega  \), \( \Omega  \) has \( C^{2} \) boundary (we say \( \partial \Omega \in  C^{2} \)), i.e.\ the domain \( \Omega  \) is locally the epigraph of a \( C^{2} \) function up to a rigid motion.

\begin{thm}[Regularity up to the boundary]
	Under the assumptions above, the function \( u \) belongs to \( H^{2}(\Omega ; \mathbb{R}^{m})  \) and moreover \( \exists c = c(\Omega , A, n) > 0 \) such that
	\begin{gather}
		\norm{u}_{H^{2}(\Omega ; \mathbb{R}^{m}) } \leq  c \left( \norm{f}_{L^{2}(\Omega ; R^{m})} + \norm{F}_{H^{1}(\Omega ; \mathbb{R}^{m \times n}) } \right).
	\end{gather}
	If both the boundary of the domain and the data are sufficiently regular the method can be iterated to obtain higher Regularity of \( u \).
\end{thm}

\begin{thm}[]
	Assume in addition to the hypothesis above that \( f \in  H^{k}(\Omega ; \mathbb{R}^{m})  \), \( F \in  H^{k+1}(\Omega ; \mathbb{R}^{m \times n})  \), \( A \in C^{k,1}(\Omega ; \mathbb{R}^{m^{2}\times n^{2}})  \) with \( \Omega  \) such that \( \partial \Omega \in C^{k+2} \). Then \( u \in H^{k+2}(\Omega ; \mathbb{R}^{m})  \)
\end{thm}

\section{Interior Regularity for Nonlinear Equations}

We see here how the Nirenberg's method is appropriate in dealing with nonlinear PDEs as those arising from Euler-Lagrange equations of non-quadratic functionals. \\
Consider \( L \in C^{2}(\mathbb{R}^{m \times n}; \mathbb{R})  \) and assume that
\begin{enumerate}[label= (\roman*)]
	\item there exists a constant \(  c>0 \) such that \( \abs{\nabla ^{2}L(\xi) } \leq c \), \( \forall \xi \in \mathbb{R}^{m \times n} \)
	\item \( L \) satisfies a uniform Legendre condition, i.e.
	      \begin{gather}
		      \sum\limits_{\alpha , \beta , i ,j }^{} \partial_{p_{j}^{\alpha }} \partial_{p_{j}^{\beta }} L(p) \xi _{i}^{\alpha } \xi _{j}^{\beta } \geq \lambda \abs{ \xi }^{2} \qquad \xi \in \mathbb{R}^{m \times n}
	      \end{gather}
	      for some \( \lambda >0  \) independent of \( p \).
\end{enumerate}
To simplify notation we set \( B_{i}^{\alpha }:= \pdv[]{L}{{p_{i}^{\alpha}}} \) and \( A_{ij}^{\alpha \beta } := \pdv[]{L}{{p_{i}^{\alpha}}}{{p_{j}^{\beta }}} \) and notice that \( A_{ij}^{\alpha \beta } \) is symmetric w.r.t the transformation \( (\alpha ,i)  \to (\beta , j) \). \\
Let \( \Omega \subset \mathbb{R}^{n} \) be an open domain and let \( u \in  H_{loc}^{1}(\Omega ; \mathbb{R}^{m})  \) be local minimizer (see later for the precise definition) of the functional
\begin{gather}
	w \mapsto \mathcal{L}(w) := \int\limits_{\Omega}^{} L(\nabla w)  \dd{x}.
\end{gather}
We will discuss the implication
\begin{gather}
	L \in  C^{\infty } \Rightarrow u \in  C^{\infty }
\end{gather}
which is strictly related ro Hilbert's XIX problem (initially posed for analytic functions of two variables).

\section{Local minimality}

We say that \( u \) is a local minimizer for \( \mathcal{L} \), if for all \( v \in  H_{loc}^{1}(\Omega ; \mathbb{R}^{m})  \) such that \( \operatorname{spt}(u-v) \subset \Omega' \ssubset \Omega  \), we have
\begin{gather}
	\int\limits_{\Omega'}^{} L(\nabla v)  \dd{x}	\geq \int\limits_{\Omega'}^{} L(\nabla u)  \dd{x}
\end{gather}
In this case one can obtain the Euler-Lagrange equation considering perturbations of \( u \) of the type \( V_{t}= u +t \varphi  \) with \(  \varphi \in C_{c}^{\infty }(\Omega ; \mathbb{R}^{m})  \) and imposing that it holds (note that \( v_{0}=u \))
\begin{gather}
	\int\limits_{\Omega}^{} L(\nabla v_{t})  \dd{x} \geq \int\limits_{\Omega}^{} L(\nabla v_{0})  \dd{x}
\end{gather}
or in other words the 1D function \( \Phi (t):= \int_{\Omega}^{} L(\nabla v_{t})  \dd{x}  \) has a local minimum at \( t=0 \), which by the regularity of \( L \) gives \( \Phi '(0)=0  \), or
\begin{gather}
	0 = \dv[]{}{t} {\left[ \int\limits_{\Omega}^{} L(\nabla u + t \nabla \varphi )  \dd{x} \right]}_{t=0} = \sum\limits_{\alpha , i}^{} \int\limits_{\Omega}^{} B_{i}^{\alpha }(\nabla u) \pdv[]{\varphi^{i}}{x_{\alpha }} \dd{x}
\end{gather}
Applying the same argument to test functions of the form \( \tau _{-h, \gamma }\varphi  \) (here \( \gamma  \) is a fixed coordinate direction corresponding to the unit vector \( e_{\gamma } \) and \( h>0 \)) we get (upon changing variables)
\begin{gather}
	\sum\limits_{\alpha , i}^{} \int\limits_{\Omega}^{} \tau _{h, \gamma } \left( B_{i}^{\alpha }(\nabla u) \right) \pdv[]{\varphi^{i}}{x_{\alpha }} \dd{x} = 0
\end{gather}
Subtracting the last two equations and dividing by \( h \)
\begin{gather}
	\sum\limits_{\alpha , i}^{} \int\limits_{\Omega}^{} \Delta _{h, \gamma } \left( B_{i}^{\alpha }(\nabla u)  \right) \pdv[]{\varphi ^{i}}{x_{\alpha }} \dd{x}	=0   \tag{EL\(_{h} \)}\label{ELh}
\end{gather}
Note that, by the regularity assumptions on \( L \) we can write that
\begin{align}
	  & B_{i}^{\alpha } (\nabla u(x+h e_{\gamma }) ) -  B_{i}^{\alpha } (\nabla u(x) )  \\
	= & \int\limits_{0}^{1} \dv[]{}{t} \left[ B_{i}^{\alpha } \left( t\nabla u(x+h e_{\gamma }) +(1-t) \nabla u(x) \right)  \right] \dd{t}  \\
	= & \sum\limits_{\beta , j}^{} \int\limits_{0}^{1} A_{ij}^{\alpha \beta } \left( t\nabla u(x+h e_{\gamma }) +(1-t) \nabla u(x) \right) \dd{t} \left( \pdv[]{u^{j}}{x_{\beta }}\/(x+h e_{\gamma }) - \pdv[]{u^{j}}{x_{\beta }}\/(x)  \right).
\end{align}
Setting for convenience
\begin{gather}
	\widetilde{A}_{ij,h}^{\alpha \beta }(x):=  \int\limits_{0}^{1} A_{ij}^{\alpha \beta } \left( t\nabla u(x+h e_{\gamma }) +(1-t) \nabla u(x) \right) \dd{t}
\end{gather}
We rewrite the~\eqref{ELh} condition as
\begin{gather}
	\sum\limits_{\alpha, \beta , i,j}^{} \int\limits_{\Omega}^{} \widetilde{A}_{ij,h}^{\alpha \beta }(x) \pdv[]{\Delta _{h, \gamma } u^{j}}{x_{\beta }}\/(x) \pdv[]{\varphi ^{i}}{x_{\alpha }}\/(x)   \dd{x} = 0
\end{gather}
Hence the function \( w = \Delta _{h, \gamma } u \) solves the system:
\begin{gather}
	- \sum\limits_{\alpha, \beta , j}^{} \partial_{x_{\alpha }} \left( \widetilde{A}_{ij,h}^{\alpha \beta } \partial_{x_{\beta }}w^{j} \right) = 0\qquad i=1,2,\ldots,n
\end{gather}
since \( \widetilde{A}_{ij,h}^{\alpha \beta } \) satisfies uniformly with respect to \( h \) both a Legendre condition and an upper bound on the \( L^{\infty } \) norm, we can apply~\eqref{CLI} to obtain that \( \exists c>0 \) independent of \( h \) such that
\begin{gather}
	\int\limits_{B_{R}(x_{0})}^{} \abs{\nabla (\Delta _{h, \gamma }u) }^{2} \dd{x} \leq  \frac{c}{R^{2}} \int\limits_{B_{2R}(x_{0})}^{} \abs{\Delta _{h, \gamma }u}^{2} \dd{x} \leq  \frac{c}{R^{2}} \int\limits_{B_{2R+h}(x_{0}) }^{} \abs{\nabla u}^{2} \dd{x} \label{009}
\end{gather}
for every \( B_{R}(x_{0}) \subset B_{2R}(x_{0}) \ssubset \Omega  \). As a result we obtain by Lemma~\ref{lemma01} that \( u \in H_{loc}^{2}(\Omega ; \mathbb{R}^{m})  \). Moreover, we have that
\begin{enumerate}[label= (\roman*)]
	\item\label{item01} \( \Delta _{h, \gamma } u \myxrightarrow[]{h \to 0} \partial_{x_{\gamma }}u \) in \( L_{loc}^{2}(\Omega ; \mathbb{R}^{m})  \) (as usual this is trivial if \( u \) is regular. In our case \( u \in  H_{loc}^{2} \) this is obtained by approximation) \\
	Notice also that since \( u \in H_{loc}^{1} \) we have \( \norm{\Delta _{h, \gamma }u}_{2} \leq c \), with, together with~\eqref{009}, gives \( \norm{\Delta _{h, \gamma }}_{H_{loc}^{1}}\leq c \). \\
	This means that, up to subsequences, \( \Delta _{h, \gamma }u \myxrightharpoonup[]{h \to 0} \partial_{x_{\gamma }}u \) weakly in \( H_{loc}^{1} \).
	\item As a result of~\ref{item01} the function \( w = \partial_{x_{\gamma }}u \) satisfies
	      \begin{gather}
		      - \sum\limits_{\alpha ,\beta ,j}^{} \partial_{x_{\alpha }} \left( A_{ij}^{\alpha \beta }(\nabla u) \partial_{x_{\beta }} w^{j} \right) = 0 \qquad i=1,2,\ldots,m
	      \end{gather}
	      in the weak sense. [It is enough to check that \( \widetilde{A}_{ij,h}^{\alpha \beta } \myxrightarrow[]{} A_{ij}^{\alpha \beta }\) in \( L_{loc}^{p}(\Omega ) \), \( \forall 1 \leq p < \infty \)]
\end{enumerate}
To solve Hilbert's XIX problem, we would like to apply a classical result by Schauder asserting that if \( w \) is a weak solution if a problem in divergence form:
\begin{gather}
	- \sum\limits_{\alpha ,\beta ,j}^{} \partial_{x_{\alpha }} \left( B_{ij}^{\alpha \beta } \partial_{x_{\beta }} w^{j} \right) = 0\qquad i = 1,2,\ldots,m
\end{gather}
them \( B \in  C^{0, \alpha }(\Omega ; \mathbb{R}^{m^{2 \times n^{2}}}) \Rightarrow w \in C^{1, \alpha }(\Omega ; R^{m}) \) which is to say \( u \in C^{2, \alpha }(\Omega ; R^{m}) \). If we know in addition that \( A_{ij}^{\alpha \beta } \) where \( C^{\infty } \), which is the case if \( L \in  C^{\infty } \), then
\begin{gather}
	A_{ij}^{\alpha \beta }(\nabla u) = B_{ij}^{\alpha \beta } \in C^{0, \alpha }(\Omega ; \mathbb{R}^{m^{2 \times n^{2}}})
\end{gather}
and the Schauder's theory would give \( w \in C^{2, \alpha } \Rightarrow u \in C^{3, \alpha } \Rightarrow B \in C^{2, \alpha } \Rightarrow \ldots \), i.e./ we can bootstrap regularity! \\
Roughly speaking, the ?? (hölderianity) result is what we need to bootstrap the argument and prove \( u \in  C^{\infty } \) if \( L \in C^{\infty } \). \\
But to do so we first need to improve the regularity of \( B(x) = A(\nabla u(x))	\), since at the moment we only know that \( A(\nabla u ) \in H_{loc}^{1}(\Omega ; \mathbb{R}^{m^{2} \times n^{2}}) \), while we would need \( A(\nabla u ) \in C^{0, \alpha }(\Omega ; \mathbb{R}^{m^{2} \times n^{2}})  \). In the case \( n=2, m \in  \mathbb{N} \) we will apply Widman's improvement of~\eqref{CLI} to prove that \( \nabla u \) is a Hölder function. The problem is much harder for \( n >2 \) and required new deep ideas. The celebrated \underline{DeGiorgi-Nash-Moser} theory solves the problem in the scalar case \( m=1 \), while for \( m >1 \) new difficulties arise.

